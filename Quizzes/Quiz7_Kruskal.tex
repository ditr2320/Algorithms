\documentclass[11pt]{article} 
\usepackage[english]{babel}
\usepackage[utf8]{inputenc}
\usepackage[margin=0.5in]{geometry}
\usepackage{amsmath}
\usepackage{amsthm}
\usepackage{amsfonts}
\usepackage{amssymb}
\usepackage[usenames,dvipsnames]{xcolor}
\usepackage{graphicx}
%\usepackage[siunitx]{circuitikz}
\usepackage{tikz}
\usepackage[colorinlistoftodos, color=orange!50]{todonotes}
\usepackage{hyperref}
\usepackage[numbers, square]{natbib}
\usepackage{fancybox}
\usepackage{epsfig}
\usepackage{soul}
\usepackage[framemethod=tikz]{mdframed}
\usepackage[shortlabels]{enumitem}
\usepackage[version=4]{mhchem}
\usepackage{multicol}

\usepackage{mathtools}
\usepackage{comment}
\usepackage{enumitem}
\usepackage[utf8]{inputenc}
\usepackage[linesnumbered,ruled,vlined]{algorithm2e}
\usepackage{listings}
\usepackage{color}
\usepackage[numbers]{natbib}
\usepackage{subfiles}
%\usepackage{tkz-berge}


\newtheorem{prop}{Proposition}[section]
\newtheorem{thm}{Theorem}[section]
\newtheorem{lemma}{Lemma}[section]
\newtheorem{cor}{Corollary}[prop]

\theoremstyle{definition}
\newtheorem{definition}{Definition}

\theoremstyle{definition}
\newtheorem{required}{Problem}

\theoremstyle{definition}
\newtheorem{ex}{Example}


\setlength{\marginparwidth}{3.4cm}
%#########################################################

%To use symbols for footnotes
\renewcommand*{\thefootnote}{\fnsymbol{footnote}}
%To change footnotes back to numbers uncomment the following line
%\renewcommand*{\thefootnote}{\arabic{footnote}}

% Enable this command to adjust line spacing for inline math equations.
% \everymath{\displaystyle}

% _______ _____ _______ _      ______ 
%|__   __|_   _|__   __| |    |  ____|
%   | |    | |    | |  | |    | |__   
%   | |    | |    | |  | |    |  __|  
%   | |   _| |_   | |  | |____| |____ 
%   |_|  |_____|  |_|  |______|______|
%%%%%%%%%%%%%%%%%%%%%%%%%%%%%%%%%%%%%%%

\title{
\normalfont \normalsize 
\textsc{CSCI 3104 Summer 2021 \\ 
Instructor: Michael Levet} \\
[10pt] 
\rule{\linewidth}{0.5pt} \\[6pt] 
\huge Quiz 7 \\
\rule{\linewidth}{2pt}  \\[10pt]
}
%\author{Your Name}
\date{}

\begin{document}
\definecolor {processblue}{cmyk}{0.96,0,0,0}
\definecolor{processred}{rgb}{200, 0, 0}
\definecolor{processgreen}{rgb}{0, 255, 0}
\DeclareGraphicsExtensions{.png}
\DeclareGraphicsExtensions{.gif}
\DeclareGraphicsExtensions{.jpg}

\maketitle


%%%%%%%%%%%%%%%%%%%%%%%%%
%%%%%%%%%%%%%%%%%%%%%%%%%%
%%%%%%%%%%FILL IN YOUR NAME%%%%%%%
%%%%%%%%%%AND STUDENT ID%%%%%%%%
%%%%%%%%%%%%%%%%%%%%%%%%%%
\noindent
Due Date \dotfill TODO \\
Name \dotfill \textbf{Didi Trifonova} \\
Student ID \dotfill \textbf{109388776} \\


\tableofcontents

\section{Instructions}
 \begin{itemize}
	\item The solutions \textbf{should be typed}, using proper mathematical notation. We cannot accept hand-written solutions. \href{http://ece.uprm.edu/~caceros/latex/introduction.pdf}{Here's a short intro to \LaTeX.}
	\item You should submit your work through the \textbf{class Canvas page} only. Please submit one PDF file, compiled using this \LaTeX \ template.
	\item You may not need a full page for your solutions; pagebreaks are there to help Gradescope automatically find where each problem is. Even if you do not attempt every problem, please submit this document with no fewer pages than the blank template (or Gradescope has issues with it).

	\item You \textbf{may not collaborate with other students}. \textbf{Copying from any source is an Honor Code violation. Furthermore, all submissions must be in your own words and reflect your understanding of the material.} If there is any confusion about this policy, it is your responsibility to clarify before the due date. 

	\item Posting to \textbf{any} service including, but not limited to Chegg, Discord, Reddit, StackExchange, etc., for help on an assignment is a violation of the Honor Code.

	\item You \textbf{must} virtually sign the Honor Code (see Section \ref{HonorCode}). Failure to do so will result in your assignment not being graded.
\end{itemize}


\section{Honor Code (Make Sure to Virtually Sign the Honor Pledge)} \label{HonorCode}

\begin{required}
On my honor, my submission reflects the following:
\begin{itemize}
\item My submission is in my own words and reflects my understanding of the material.
\item I have not collaborated with any other person.
\item I have not posted to external services including, but not limited to Chegg, Discord, Reddit, StackExchange, etc.
\item I have neither copied nor provided others solutions they can copy.
\end{itemize}

\noindent In the specified region below, clearly indicate that you have upheld and agree to the Honor Code. Then type your name.
\end{required}

\begin{proof}[Honor Pledge]
I have upheld and agree to the Honor Code - Didi Trifonova
\end{proof}


\newpage
\section{Standard 7- Kruskal's Algorithm}

\subsection{Problem \ref{Safe1}}
\begin{required} \label{Safe1}
Consider the following graph $G(V, E, w)$. Clearly indicate the order in which Kruskal's algorithm adds the edges to the minimum-weight spanning tree. You may simply list the order of the edges; it is not necessary to exhibit the state of the algorithm (i.e., the disjoint-set data structure) at each iteration.
\begin{center}
\begin {tikzpicture}[semithick]
\tikzstyle{blue}=[circle ,top color =white , bottom color = processblue!20 ,draw,processblue , text=blue , minimum width =1 cm];
\tikzstyle{red}=[circle ,top color =white , bottom color = processred!20 ,draw, processred , text=blue , minimum width =1 cm];
\tikzstyle{green}=[circle ,top color =white , bottom color = processgreen!20 ,draw, processgreen , text=blue , minimum width =1 cm];

	\node[blue] (A) {$A$};
	\node[blue] (B) [above right = of A] {$B$};
	\node[blue] (C) [below right = of A] {$C$};
	\node[blue] (D) [right = of B] {$D$};
	\node[blue] (E) [right = of C] {$E$};
	\node[blue] (F) [below right = of D] {$F$};
	
	\path (A) edge node[above] {$2$} (B);
	\draw (A) edge node[right] {$5$} (C);
	\path (B) edge node[left] {$8$} (C);
	\path (B) edge node[above] {$5$} (D);
	\path (C) edge node[above] {$4$} (E);
	\draw (E) edge node[right] {$2$} (D);
	\path (D) edge node[above] {$7$} (F);
	\draw (E) edge node[below] {$5$} (F);
	\end{tikzpicture}  
\end{center}
\end{required}


\begin{proof}[Answer]
%You may type your answer here if you wish.
Edges sorted from least cost to most: \\
BA(2), DE(2), CE(4), AC(5), BD(5), EF(5), DF(7),BC(8) 
\begin{itemize}
    \item Add BA because it is not useless
    \item Add DE because it is not useless
    \item Add CE because it is not useless
    \item Add AC because it is not useless
    \item Do not add BD because it is useless 
    \item Add EF because it is not useless
    \item Do not add DF because it is useless 
    \item Do not add BC because it is useless 
\end{itemize}
Final order: BA,DE,CE,AC,EF
\end{proof}

% Either type your answer in below, or uncomment the \includegraphics command
% and use it to insert an approprate image. Try experimenting with the scale 
% 0.9 the width option to resize your image if necessary.

%\includegraphics[width=0.9\textwidth]{solution.jpg}


%%%%%%%%%%%%%%%%%%%%%%%%%%%%%%%%%%%%%%%%%%%%%%%%%%
\end{document} % NOTHING AFTER THIS LINE IS PART OF THE DOCUMENT



